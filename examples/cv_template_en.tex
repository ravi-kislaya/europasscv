\documentclass[helvetica,english,logo,notitle,totpages,utf8]{europecv2013}
\usepackage{graphicx}
\usepackage[a4paper,top=1.2cm,left=1.2cm,right=1.2cm,bottom=2.5cm]{geometry}
\usepackage[english]{babel}
\usepackage[T1]{fontenc}
\usepackage{comment}

\usepackage{natbib}
\usepackage{bibentry}
\nobibliography*

\usepackage{ifthen}
\newboolean{academic}
\setboolean{academic}{false}
\newboolean{numbereditems}
\setboolean{numbereditems}{false}
\newcounter{numbercount}
\setcounter{numbercount}{3}
\newcommand{\addnumber}{
\ifthenelse{\boolean{academic}\and\boolean{numbereditems}}{{\bfseries{[Attachment A\thenumbercount]}}\addtocounter{numbercount}{1}}{}
}

\usepackage{pgf,pgffor}

% Your list of publications (by bibliography label)

% papers
\newcommand{\papers}{book1,article1,article2}

% technical reports
\newcommand{\technicalreports}{techreport1,techreport2,techreport3}

% journal reviews
\newcommand{\journalreviews}{
	{International Journal of Latex Guru (IJLG), 2016}, % Technical Program Committee
	{Journal of Markup Languages (JML), 2015} % Designated Reviewer
}

% conference reviews
\newcommand{\conferencereviews}{
	{International Conference on Computer Systems 2014 (ICCS 2014), 2014}, % Designated Reviewer
	{International Conference on Programming Languages 2015 (ICPL 2015), 2015} % Designated Reviewer
}

% Automatically computed values

% papers count
\newcounter{paperscount}
\setcounter{paperscount}{0}
\foreach \x [count=\xi] in \papers {\addtocounter{paperscount}{1}}

% technical reports count
\newcounter{technicalreportscount}
\setcounter{technicalreportscount}{0}
\foreach \x [count=\xi] in \technicalreports {\addtocounter{technicalreportscount}{1}}

% journal reviews count
\newcounter{journalreviewscount}
\setcounter{journalreviewscount}{0}
\foreach \x [count=\xi] in \journalreviews {\addtocounter{journalreviewscount}{1}}

% conference reviews count
\newcounter{conferencereviewscount}
\setcounter{conferencereviewscount}{0}
\foreach \x [count=\xi] in \conferencereviews {\addtocounter{conferencereviewscount}{1}}

% total reviews count
\newcounter{totalreviewscount}
\setcounter{totalreviewscount}{\thejournalreviewscount}
\addtocounter{totalreviewscount}{\theconferencereviewscount}

% personal information
\ecvname{Kislaya Ravi}
\ecvaddress{0337, Grassmeirestrasse 17, Munich, 80805 (Germany)}
\ecvtelephone[+49 179 4111486]{+49 179 4111486}
\ecvemail{kislayaravi24@gmail.com}
%\ecvhomepage{\href{http://www.mariorossisamplewebsite.com}{www.mariorossisamplewebsite.com}}
\ecvlinkedin{\href{http://www.linkedin.com/in/kislaya-ravi}{linkedin.com/in/kislaya-ravi}}
%\ecvgender{Male}
\ecvdateofbirth{24 February, 1991}
\ecvnationality{Indian}

\ecvfootnote{}
\ecvbeforepicture{\raggedleft}
%\ecvqrcode[width=3cm]{../img/qrcode}
%\ecvafterqrcode{\ecvspace{-2.5cm}}
\ecvpicture[height=3cm]{../img/photo.jpg}

\begin{document}

\selectlanguage{english}

\begin{europecv}

\thispagestyle{plain}

\ecvpersonalinfo%[10pt]

\ecvposition{Job applied for}{Working Student}

\ecvsection{Education and training}

\ecveducation{Oct 2016 -- March 2019}{Master in Science, Computational Science and Engineering}
{Technical University, Munich, Germany}{
    \begin{itemize}
        \item Multidisciplinary program with confluence of Computer Science, Applied Mathematics and an application field of once choice
        \item Mathematical modelling, Numerical analysis, Efficient Algorithms, Computer Architecture, Software Design and Implementation, Validation, and Visualization of results
        \item Master's Thesis: Neural Network Hyperparameter Optimization using SNOWPAC
            \begin{itemize}
                \item[-] Developed a mixed integer trust-region optimization tool in C++ to optimize non-linear constrained mixed-integer optimization problems
                \item[-] Used the developed code to optimize hyperparameters of a neural network and compare it with existing methods
            \end{itemize}
    \end{itemize}
}{}

\ecveducations{June 2009 -- June 2014}{Bachelor of Technology, Mechanical Engineering} 
{Indian Institute of Technology, Banaras Hindu University, Varanasi, India}{
\begin{itemize}
    \item Both master and bachelor degreee awarded as part of Integrated dual degree course
    \item Basics of mechanical engineering like machine design, fluid dynamics, solid mechanics, production engineering, operational methods, etc.
    \item Masters Thesis: Axi-symmetric dynamic response of buried orthotropic cylindrical shells due to compressive wave using Finite Difference techniques. \par
            - Implemented code in Matlab to solve shell equation and studied the behavior of lifelines like sewage pipes, water pipelines, etc. during earthquakes
\end{itemize}
}
{Master of Technology, Machine Design}


\ecvsection{Work experience}

\ecvworkexperience{Aug 2017 -- Sep 2018}{Research Assistant}
{Engineering Risk and Analysis}
{Technical University Munich, Germany}
{
    - Project BAYES: Bayesian updating of engineering models with spatially variable properties \par
    - Generated random field in C++ for given mean and covariance operator and solved corresponding elliptical and parabolic PDE using FEniCS \par
    - Made API in Python by wrapping C++ code to be used for inverse problem solution
}

\ecvworkexperience{July 2014 -- July 2016}{Assistant Manager}{Maruti Suzuki India Limited, Gurugram, India}{
    - Design and Development of Exterior Trims for New Model Development  \par
    - Co-ordinated and managed external vendors for timely development \par
    - Understood various design standards, techniques and pipelines in automobile sector
}

\ecvsection{Relevant Projects}

\ecveducation {June 2017 -- July 2017} {Lattice Boltzmann Method} {Technical University Munich}{
    - Topic: Simulation of Blood Flow through Aorta using Lattice Boltzmann Method \par
    - Implemented Lattice Boltzmann method in C++ for unstructured grid and parallelised the code using MPI \par
    - Decomposed the unstructured domain for proper load balancing \par
    - Got acquainted with various techniques used in parallel programming and High Performance Computing
}

%\ecveducation {September 2017 -- October 2017} {Computational Stochastic Dynamics} {Technical University Munich}{
    %- Topic: Solution of Particle Dispersion under Divergence Free Flow using Monte Carlo, Ito’s Calculus, Path Integral and Equivalent Linearization Methods \par
    %- Implemented and compared various method to study dispersion of pollutants in coastal region in Matlab \par
    %- Got acquainted with various techniques used to model and approximate the stochastic processes
%}

\ecveducation {September 2018 -- October 2018} {Blind Deconvolution of Blurry Images using CUDA} {Technical University Munich}{
    - Topic: GPU Parallelisation of Total Variation Blind Deconvolution using CUDA \par
    - Constructing sharp image from blurry image without prior knowledge of degradation \par
    - Implemented and tuned the code in C++ CUDA framework \par
    - Learned basics of OpenCV, CUDA, profiling and optimization techniques in CUDA
}

\ecvsection{Computer Skills}

\ecvitem[10pt]{Programming Language}{
    C, C++, Python, Matlab, Bash Scripting
}

\ecvitem[10pt]{Frameworks/ Tools}{
    PyTorch, CUDA, OpenCV, CMake, Git
}

\ecvitem[10pt]{Libraries}{
    ARPACK++, GNU Plot, BOOST, FEniCS, MPI, OpenMP, METIS
}

\ecvitem[10pt]{Softwares ans Operating Systems}{
    \LaTeX , Microsoft Office, Paraview, Windows, Linux
}

\ecvsection{Scientific Publications}

\ecvitem[10pt]{  }{
    Uttam, Vishad, Jain, Nitin, and Ravi, Kislaya. Phenomenon of Corrosion in Chrome Plated ABS Parts. No. 2016-28-0066. SAE Technical Paper, 2016.
}
%TODO: Work on this later
%\ecvitem{\ifthenelse{\boolean{academic}}{Scientific Papers}{}}{\begin{itemize}
%\item \foreach \x [count=\xi] in \papers {\ifnum\xi=1 \bibentry{\x} \addnumber \fi} % first item
%\foreach \x [count=\xi] in \papers {\ifnum\xi=1 \else \item \bibentry{\x} \addnumber \fi} % remaining items
%\end{itemize}
%}

%\ifthenelse{\boolean{academic}}{\ecvitem{Technical Reports}{\begin{itemize}
%\foreach \x [count=\xi] in \technicalreports {\item \bibentry{\x} \addnumber\ }
%\end{itemize}}}{}

%\ifthenelse{\boolean{academic}}{\ecvitem{Review Activities}{

%\vspace{0.3em}

%\begin{itemize}
%\foreach \x [count=\xi] in \journalreviews {\item {\x} \addnumber}
%\foreach \x [count=\xi] in \conferencereviews {\item {\x} \addnumber}
%\foreach \x [count=\xi] in \technicalreports {\item {\x} \addnumber}
%\end{itemize}}}{}

\ecvsection{Personal skills}

\ecvmothertongue[20pt]{Hindi}
\vspace{-2em}
\ecvlanguageheader
\ecvlanguage{English}{C2}{C2}{C2}{C2}{C2}
\ecvlastlanguage{German}{A2}{A2}{A2}{A2}{A2}
\ecvlanguagefooter[10pt]
\vspace{-2em}
\ecvitem[10pt]{Organisational / managerial skills}{
- Team Leader: Lead the team which represented IIT (BHU), Varanasi in SAE Efficycle 2012, a national level three wheeler efficient vehicle design competition \par
- SAE Events Organization: Organized various workshops \par
- Training and Placement Representative Mechanical IDD, 2012
}

\ecvitem[10pt]{Teaching Experience}{
    Tutorial of Mechanical Measurements at IIT (BHU), Varanasi in 2012
}

\ecvitem[10pt]{Hobbies}{Forr{\'o} Dancing, Table Tennis, Jogging}

\end{europecv}

\bibliographystyle{plainnat}
\newsavebox\mytempbib
\savebox\mytempbib{\parbox{\textwidth}{\bibliography{bibliography}}}

\end{document} 
